Entre os algoritmos analisados, foi possível perceber claramente uma superioridade entre os algoritmos de busca com informação nos aspectos de tempo e estados expandidos. Isso demonstra a clara vantagem de se possuir algum tipo de conhecimento sobre o espaço em que se está fazendo a busca. Também foi possível observar como a otimalidade do algoritmo A* depende da heurística utilizada. Com uma heurística apropriada, o A* se mostra o algoritmo superior.

Os algoritmos de busca sem informação tiveram um desempenho pior em número de estados expandidos e tempo de execução. O UCS se manteve relativamente próximo ao desempenho dos outros algoritmos, porém o desempenho do algoritmo IDS foi drasticamente inferior até mesmo ao UCS. Seu tempo de execução tornou inviável o teste para distâncias muito grandes. A complexidade de tempo de um DFS normal é $O(b^m)$, onde $m$ é o tamanho máximo de um caminho, o que comparado a complexidade do UCS, $O(b^d)$, onde $d$ é a profundidade da solução mais rasa, é pior. O IDS ainda deve executar um número de iterações igual à profundidade da solução, ou no caso da implementação nesse trabalho, ao custo da solução. Essas diversas iterações tornam o IDS ainda mais lento. Sua vantagem está somente na melhor complexidade de espaço.